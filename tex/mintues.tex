\documentclass[dvipdfmx]{jsarticle}

\begin{document}
プログラミング方法論 - 第3回 T2 議事録\\
\\
<会議名> プログラミング方法論(第3回) 進捗報告\\
<日時> 2025年5月15日(木)13:25 - 15:15\\
<場所> 工学部4号館303\\
<参加者(敬称略)> 乃村\\
T1:高尾,藤原,重見,平田\\
T2:大槻,片山,田中,鄭\\
T3:内田,吉村,下稲,飯田\\

<資料> scrapbox参照 (url: https://scrapbox.io/2025PM2/)\\

\section{議論概要}
\subsection{MCPの目的について}
\begin{enumerate}
    \item 本来概要とは,それ以降の説明と対応してあるべきである.したがって,ただ抽象的な内容を書くのではなく,今回のプロジェクトで行ったことから逆算的に考えて,図などを作成し,概要を説明するべきである.\hfill{(乃村)}\\
    \item そのために,他のメンバーの作成内容を把握して,概要を作成するべきである.\hfill{(乃村)}
    \item RAGとどう違うのか,また,RAGのどのような問題があってMCPが生まれたのか示してほしい.   \hfill{(乃村)}\\
\end{enumerate}

\subsection{MCPの構成についての調査}
\begin{enumerate}
    \item Anthropic社のHPから引用したアーキテクチャの図はわかりにくい.\hfill{(乃村)}\\
    \item 例えば,アーキテクチャの図内で,"client 1"は何なのか明示したり,具体的なツール名を用いたりしたほうがわかりやすい.\hfill{(乃村)}
    \begin{enumerate}
        \item "client 1"などが何を指すのか具体的な名前を明示する.\hfill{(田中)}\\
    \end{enumerate}
    \item クライアントが実際どのようなものなのか?\hfill{(乃村)}
    \begin{enumerate}
        \item 現在はホストとサーバの仲介役という認識である.\hfill{(田中)}\\
    \end{enumerate}
    \item MCPは具体的なコードを書いてみたからこそわかることもあるため,実装からわかったことを盛り込んだ形で作成することが望ましい.\hfill{(乃村)}\\
    \item シーケンス図はホスト,クライアント,サーバの関係を示すより,LLM,クライアント/ホスト,サーバの関係および間でやりとりされるプロンプト例があると良い.\hfill{(乃村)}\\
\end{enumerate}

\subsection{MCPのプロトコルに関する調査}
\begin{enumerate}
    \item クライアントとサーバ間の通信の流れを示した図の説明は重要ではない.\hfill{(乃村)}\\
    \begin{enumerate}
        \item LLM/Client/Serverのやりとりの説明が重要である.\hfill{(乃村)}\\
    \end{enumerate}
    \item 天気を聞くシナリオだと,tools要求に対しての返答が単純(一つのツールしか返ってこない)なため,複数のツールが返答されるようなシナリオを用いたほうが良い\hfill{(乃村)}\\
    \item MCPのユニークな部分を主張してほしい\hfill{(乃村)}\\
    \begin{enumerate}
        \item 例えば,MCPはLLMを用いることで自然言語で指示することができることや,文法はJSONで指定するだけでシンプルな構造であるなど.\hfill{(乃村)}\\
        \item RAGとの違いも示すと良い.RAGだとできない例が示してあると良い.\hfill{(乃村)}\\
    \end{enumerate}
\end{enumerate}

\subsection{MCPを取り巻く環境の変遷}
\begin{enumerate}
    \item この節では,MCPの歴史について述べ,今後の展望や課題点にも触れる.また,RAGとの違いについても触れる.\hfill{(片山)}
    \begin{enumerate}
        \item MCP関連の話題や最近の動向をまとめる試みは面白いと思う.また,RAGとの違いについての説明があるのも良い.\hfill{(乃村)}\\
    \end{enumerate}
\end{enumerate}

\section{次回予定}
<内容> スライドチェック1\\
<日時> 2025年5月29日(木)13:25 - 15:15\\
<場所>工学部4号館303\\
\end{document}
